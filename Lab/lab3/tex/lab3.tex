%!TEX encoding = UTF-8 Unicode
%!TEX program = xelatex

\documentclass{article}

\title{ICS Lab 3}
\author{@taoky}
\date{\today}

\usepackage{amsmath}
\usepackage{amsfonts}
\usepackage{enumerate}
\usepackage{graphicx}
\usepackage[super]{nth}
\usepackage{xcolor}
\usepackage[toc,page]{appendix}
\usepackage[hyphens,spaces]{url}
\usepackage{listings,lstautogobble}

\bibliographystyle{plain}

\lstdefinelanguage[lc3]{Assembler}{
	morekeywords={[1]ADD, AND, BRn, BRzp, BRz, BRnp, BRp, BRnz, BR, BRnzp, JMP, JSR, JSRR, LD, LDI, LDR, LEA, NOT, RET,%
		RTI, ST, STI, STR, TRAP, HALT},
	morekeywords={[2].ORIG, .FILL, .BLKW, .STRINGZ, .END},
	morekeywords={[3]R0, R1, R2, R3, R4, R5, R6, R7},
	alsoletter={x\#},
	morestring=[b]",
	morestring=[b]',
	morecomment=[l];,
}[strings,comments,keywords]

\lstdefinelanguage
   [x64]{Assembler}     % add a "x64" dialect of Assembler
   [x86masm]{Assembler} % based on the "x86masm" dialect
   % with these extra keywords:
   {morekeywords={CDQE,CQO,CMPSQ,CMPXCHG16B,JRCXZ,LODSQ,MOVSXD, %
                  POPFQ,PUSHFQ,SCASQ,STOSQ,IRETQ,RDTSCP,SWAPGS, %
                  rax,rdx,rcx,rbx,rsi,rdi,rsp,rbp, %
                  r8,r8d,r8w,r8b,r9,r9d,r9w,r9b, %
                  r10,r10d,r10w,r10b,r11,r11d,r11w,r11b, %
                  r12,r12d,r12w,r12b,r13,r13d,r13w,r13b, %
                  r14,r14d,r14w,r14b,r15,r15d,r15w,r15b}} % etc.

\lstset{
	language=[lc3]{Assembler},
	basicstyle=\ttfamily,
	autogobble=true,
	breaklines=true,
	numbers=left,
	postbreak=\mbox{\textcolor{red}{$\hookrightarrow$}\space},
	captionpos=b,
}

\begin{document}
	\maketitle
	\section{Introduction}
	In this lab, the task is a bit simple: compile the C code in Lab 2 with a 3rd-party LC-3 C compiler to LC-3 object file, and compare the differences between the generated result and that in our Lab 2.
	
	\section{Compiling}
	We use the LC-3 compiler (LCC) officially provided at the website of our ICS book\cite{lcc}, and build it on Linux with \texttt{gcc}, as there's something wrong when compiling that on macOS with \texttt{clang}. 
	
	The C code modification is required. We change the C code in Lab 2 to the following.
	
	\begin{lstlisting}[language=c, caption={The compiled C code}]
	typedef short i16;
	typedef unsigned short u16;
	i16 func(i16 n, i16 a, i16 b, i16 c, i16 d, i16 e, i16 f)
	{ //Lots of arguments, hah?
	    i16 t = getchar() - '0' + a + b + c + d + e + f;
	    if (n > 1)
	    {
	        i16 x = func(n - 1, a, b, c, d, e, f);
	        i16 y = func(n - 2, a, b, c, d, e, f);
	        return x + y + t - 1;
	    }
	    else
	    {
	        return t;
	    }
	}
	i16 main(void)
	{
	    i16 n = getchar() - '0';
	    return func(n, 0, 0, 0, 0, 0, 0);
	}
	\end{lstlisting}
	
	The \texttt{\_\_start()} function is removed, and \texttt{GETC()} is modified to \texttt{getchar()}, as there's no ``\texttt{GETC}" in C standard library. Use \texttt{lcc} just like using \texttt{gcc} or other command-line compilers.
	
	With file name as its only argument, the compiled object file is saved as \texttt{a.obj}. Set a breakpoint at \texttt{HALT}, and try running it with \texttt{lc3sim}, and... Where is our result?
	
	\begin{lstlisting}[language=sh, caption={Where is the result?}]
	(lc3sim) c
	98765432111111111111111<skipping many 1s>The LC-3 hit a breakpoint...
	PC=x300B IR=xC1C0 PSR=x0800 (NEGATIVE)
	R0=x30A7 R1=x7FFF R2=x0000 R3=x0001 R4=x30EE R5=xEFFF R6=xEFFE R7=x300B 
	B                  x300B xF025 HALT 
	\end{lstlisting}
	
	Here, I expect a result of \texttt{x0053}, and none of our registers is the value. We know that in LC-3 system, \texttt{R6} is usually the stack pointer. So what's around \texttt{R6}?
	
	\begin{lstlisting}[language=sh, caption={Contents around \texttt{R6}}]
	(lc3sim) list xeffe
	                   xEFFE x0053 .FILL 'S'
	                   xEFFF x0000 NOP   
	                   xF000 x0000 NOP   
	                   xF001 x0000 NOP   
	                   xF002 x0000 NOP   
	                   xF003 x0000 NOP   
	                   xF004 x0000 NOP   
	                   xF005 x0000 NOP   
	                   xF006 x0000 NOP   
	                   xF007 x0000 NOP   
	                   xF008 x0000 NOP   
	                   xF009 x0000 NOP   
	                   xF00A x0000 NOP   
	                   xF00B x0000 NOP   
	                   xF00C x0000 NOP   
	                   xF00D x0000 NOP   
	                   xF00E x0000 NOP   
	                   xF00F x0000 NOP   
	                   xF010 x0000 NOP   
	                   xF011 x0000 NOP
	\end{lstlisting}
	
	Well, the result is at the top of our stack. And now there's only one step before finishing the program: add an instruction to set \texttt{R0} to the right value.
	
	Add \texttt{LDR R0, R6, \#0} before \texttt{HALT}, and it's don... Wait, the behavior of our program is still slightly different from the requirements of Lab 2: \texttt{getchar()} and the \texttt{GETC} trap is different that \texttt{getchar()} gets characters echoed, but \texttt{GETC} doesn't.
	
	The fix is simple: remove or comment the \texttt{OUT} trap in \texttt{lc3\_getchar} subroutine. It's really done now.
	
	\section{Go Deeper}
	
	What's inside the compiled object file? It's required to make comparisons between the compiled result and our hand-written file. What's different here?
	
	\subsection{The Initialization Part}
	
	In Lab 2, I set register \texttt{R6} to the stack base address, and set \texttt{R5} for checking stack overflow, other registers are set to zero, and then \texttt{JSR} to the subroutine \texttt{MAIN}. The stack base address, stack overflow check value (the negative of \texttt{xEE00}), and stack overflow error message are written after the label \texttt{GLOBAL}.
	
	The C compiler, however, does the initialization part in a different way. In the \texttt{INIT\_CODE} label, it does something to our lovely \texttt{R6} and \texttt{R5}. I know that in the book\cite[Chapter~14]{ics}, the \texttt{R5} is the frame pointer, rather than a ``error-checking" value.
	
	As our program starts at \texttt{x3000}, it load the effective address to \texttt{R6} first, setting it to \texttt{x3000}, then after some \texttt{ADD}s, \texttt{R6} and \texttt{R5} are both set to \texttt{xEFFF}.
	
	Then, \texttt{R4} is set to the value of \texttt{GLOBAL\_DATA\_POINTER}, which contains \\ \texttt{GLOBAL\_DATA\_START}, indicating the start address of data stored in this program. And \texttt{R7} is set to the address of \texttt{main} subroutine, then \texttt{JSRR} to \texttt{R7}, staring our \texttt{main()}.
	
	\begin{lstlisting}[caption={The contents of \texttt{GLOBAL\_DATA\_START}}]
	GLOBAL_DATA_START
	func .FILL lc3_func
	L3_lc3 .FILL lc3_L3_lc3
	L1_lc3 .FILL lc3_L1_lc3
	L8_lc3 .FILL lc3_L8_lc3
	getchar .FILL lc3_getchar
	L9_lc3 .FILL #0
	L6_lc3 .FILL #2
	L5_lc3 .FILL #1
	L2_lc3 .FILL #48
	\end{lstlisting}
	
	The function address other than \texttt{main()} and some constants are stored in the program.
	
	In one word, the difference lays that the compiled program has a extra function address table inside, and takes registers from \texttt{R4} to \texttt{R7} for use. Here, the usage of \texttt{R4} and \texttt{R5} is completely different from what I did in Lab 2.
	
	\subsection{Calling Conventions, Prolog and Epilog}
	Obviously calling conventions are different here. In Lab 2, I didn't take the existing calling conventions in book quite seriously, and for the sake of convenience and speed, I change a lot of things in calling conventions. In Lab 2, my calling conventions require caller-save and callee-cleanup. The arguments are pushed to stack from right to left, and return value is directly stored in \texttt{R0}. \texttt{R5} is not used for frame pointer. For more details, see \cite{lab2-tky}.
	
	What's about our compiler? Take the simplest function \texttt{lc3\_getchar} as an example.
	
	\begin{lstlisting}[caption={LC-3 implementation of \texttt{getchar()}}]
	; char getchar(void)
	
	lc3_getchar
	
	STR R7, R6, #-3
	STR R0, R6, #-2
	GETC
	; OUT
	STR R0, R6, #-1
	LDR R0, R6, #-2
	LDR R7, R6, #-3
	ADD R6, R6, #-1
	RET
	\end{lstlisting}
	
	This function:
	\begin{itemize}
	\item Store the value of \texttt{R7} (the return address) and \texttt{R0} (the value it will modify) in stack. The stack pointer is not updated, as we know that \texttt{GETC} is a good trap routine, and it won't do anything bad to our stack.
	\item After it finishes its work, it stores the return value in stack, restores \texttt{R0} and \texttt{R7} to their original values, and updates the stack pointer.
	\item It has no local variables, as it doesn't do anything to \texttt{R5}.
	\end{itemize}
	
	Now we can briefly know what's happening. However, we have to analyze functions more deeply. Let's take a look at \texttt{main()}!
	
	\begin{lstlisting}[caption={The beginning of \texttt{main}}]
	main
	ADD R6, R6, #-2
	STR R7, R6, #0
	ADD R6, R6, #-1
	STR R5, R6, #0
	ADD R5, R6, #-1
	\end{lstlisting}
	
	At the very beginning, it pushes 3 values to stack: the return value, which will be filled after the function finishes its task; old \texttt{R7} and old \texttt{R5}.
	
	Then preparing to call subroutine \texttt{getchar()}.
	
	\begin{lstlisting}[caption={Calling \texttt{getchar()}}]
	ADD R6, R6, #-1
	ADD R0, R4, #4
	LDR R0, R0, #0
	jsrr R0
	\end{lstlisting}
	
	It moves the stack pointer further for the local variable \texttt{n}, and prepares \texttt{R0} to the address of \texttt{lc3\_getchar}, then \texttt{JSRR} to that subroutine.
	
	What happens after it returns to \texttt{main}? At that time, the top of stack is the return value of \texttt{getchar()}.
	
	\begin{lstlisting}[caption={Things after \texttt{getchar()}, with comments}]
	LDR R7, R6, #0 ; R7 = return value
	ADD R6, R6, #1 ; pop return value
	ADD R3, R4, #8 ; load table
	ldr R3, R3, #0 ; R3 = 48 ('0')
	ADD R6, R6, #-1 ; prepare pushing
	STR R0, R6, #0 ; push R0 (the address of lc3_getchar)
	ADD R6, R6, #-1 ; prepare pushing
	STR R3, R6, #0 ; push 48
	NOT R3, R3
	ADD R3, R3, #1 ; R3 = -48
	ADD R0, R7, R3 ; R0 = GETC() - 48
	LDR R3, R6, #0 ; R3 = 48 (reload R3)
	ADD R6, R6, #1 ; pop 48
	ADD R7, R0, #0 ; R7 = R0 = GETC() - 48
	LDR R0, R6, #0 ; R0 = the address of lc3_getchar
	ADD R6, R6, #1 ; pop 
	str R7, R5, #0 ; store R7's result
	\end{lstlisting}
	
	It handles with the return value on stack, pops that, calculates the value minus 48, and stores as a local variable.
	
	At this time, \texttt{getchar()} is a function having no arguments. What's about \texttt{func()}?
	
	\begin{lstlisting}[caption={Calling \texttt{func()}, a function with many arguments}]
	ADD R7, R4, #5 ; load table
	ldr R7, R7, #0 ; R7 = 0
	ADD R6, R6, #-1
	STR R7, R6, #0 ; stores 7th arg
	ADD R6, R6, #-1
	STR R7, R6, #0 ; stores 6th arg
	ADD R6, R6, #-1
	STR R7, R6, #0 ; stores 5th arg
	ADD R6, R6, #-1
	STR R7, R6, #0 ; stores 4th arg
	ADD R6, R6, #-1
	STR R7, R6, #0 ; stores 3rd arg
	ADD R6, R6, #-1
	STR R7, R6, #0 ; stores 2nd arg
	ldr R7, R5, #0 ; load 1st arg = GETC() - 48
	ADD R6, R6, #-1
	STR R7, R6, #0 ; stores 1st arg
	ADD R0, R4, #0 ; load table
	LDR R0, R0, #0 ; R0 = address of func
	jsrr R0
	\end{lstlisting}
	
	It stores arguments from right to left on stack, the same as my implementation. After calling \texttt{func()}, it takes the return value on stack in \texttt{R7}, and pops that -- the same as what happens after \texttt{getchar()}.
	
	The final step is the clean-up. When debugging the program, we can notice that the value of \texttt{R6} is not restored, which means that the arguments of \texttt{func()} are still in stack. But \texttt{R5} is restored.
	
	\begin{lstlisting}[caption={Before returning from \texttt{main()}}]
	lc3_L8_lc3
	STR R7, R5, #3 ; save return value as the return value
	ADD R6, R5, #1 ; pop local variables
	LDR R5, R6, #0 ; restore the R5 of caller
	ADD R6, R6, #1 ; pop
	LDR R7, R6, #0 ; restore the R7 (return address) of caller
	ADD R6, R6, #1 ; pop
	RET ; return
	\end{lstlisting}
	
	So, local variables are popped, and the values of \texttt{R5} and \texttt{R7} are restored to their caller's (initialization function) values.
	
	And now we can make conclusions here about \texttt{LCC}'s calling conventions.
	
	\begin{itemize}
	\item Function arguments are pushed on the stack in the right-to-left order, same as my implementation.
	\item Before jumping into a subroutine (function), the caller should:
	\subitem Save local variables to stack.
	\subitem Push arguments to stack.
	\item After jumping into a subroutine (function), the callee should:
	\subitem Push the return value and caller's frame pointer to stack.
	\subitem Update frame pointer according to the stack pointer.
	\subitem Doing work.
	\subitem After others are done, save return value on stack, pop local variables, restore frame pointer and return address to \texttt{R5} and \texttt{R7}, and pop them, then return.
	\item After it’s returned from the subroutine (function), the caller should:
	\subitem Save return value on stack, and pop that.
	\subitem Pop arguments according to \texttt{R5}.
	\end{itemize}
	
	\subsection{Things After \texttt{main()}}
	After \texttt{main()}'s execution, it directly \texttt{HALT} the system. To get the result in \texttt{R0}, as is mentioned before, a \texttt{LDR R0, R6, \#0} is required.
	
	\subsection{Some More Details about Stack}
	The stack in the program goes from \texttt{xEFFE} (\texttt{xEFFF} stores nothing here). When pushing values, the stack pointer is decremented. There's no error handling about stack overflow.
	
	\section{Real-life Calling Conventions and ABI}
	\subsection{About System V AMD64 ABI}
	System V AMD64 ABI is mainly used on Unix-like OS like Linux on AMD64 environment. The official document is \cite[The x86-64 psABI version 1.0]{amd64-abi}.
	
	What is it like?
	
	The document is quite long and tedious, so I found a simple document\cite{osdev-abi} about that ABI. Let's go.
	\subsubsection{What's System V?}
	UNIX System V is a UNIX version, which is quite success and has many features like \texttt{/etc/init.d} which are widely used in later Unix-like operating systems.
	
	\subsubsection{Executable File Format}
	ELF (Executable and Linkable Format) is used as the standard executable file format in System V ABI. In the UNIX world, a.out format appears first, then it was replaced by COFF format, whose variant - PE, acts as the standard executable format in Windows. And then COFF is replaced by ELF now. And, well, the Unix-like macOS takes Mach-O as its standard. This is one of the reasons why a simple C program compiled in macOS cannot be taken to Linux directly, which means they are not binary compatible.
	
	I'm not going to take a deep look at the ELF format, as it is too far away from this lab. However, I think taking a tender glance at that is OK.
	
	\begin{lstlisting}[language=sh, caption={Taking a look at the ELF file \texttt{/bin/ls}}]
	[tao@tao-linux-vmware lab3]$ readelf -h /bin/ls
	ELF Header:
	  Magic:   7f 45 4c 46 02 01 01 00 00 00 00 00 00 00 00 00 
	  Class:                             ELF64
	  Data:                              2's complement, little endian
	  Version:                           1 (current)
	  OS/ABI:                            UNIX - System V
	  ABI Version:                       0
	  Type:                              DYN (Shared object file)
	  Machine:                           Advanced Micro Devices X86-64
	  Version:                           0x1
	  Entry point address:               0x5ae0
	  Start of program headers:          64 (bytes into file)
	  Start of section headers:          136040 (bytes into file)
	  Flags:                             0x0
	  Size of this header:               64 (bytes)
	  Size of program headers:           56 (bytes)
	  Number of program headers:         11
	  Size of section headers:           64 (bytes)
	  Number of section headers:         25
	  Section header string table index: 24
	
	\end{lstlisting}
	
	By using \texttt{readelf}, we can have a look at the header of this ELF file. We can notice that it uses \texttt{UNIX - System V} ABI, and it runs on AMD64 machine.
	
	\subsubsection{Calling Conventions}
	
	The stack grows downwards in System V AMD64 ABI. The first 6 integer or pointer arguments are stored in registers: \texttt{rdi}, \texttt{rsi}, \texttt{rdx}, \texttt{rcx}, \texttt{r8}, \texttt{r9}, while further arguments are pushed to stack reversely. Caller should clean-up the stack. Floating point numbers use XMM registers, which I won't discuss later.
	
	The register \texttt{rsp} acts as the stack pointer, and \texttt{rbp} acts as the frame pointer. However, in my LC-3 calling conventions, the frame pointer is not used. In System V AMD64 ABI, the frame pointer is optional. \texttt{gcc} has an argument \texttt{-fomit-frame-pointer}, which can adjust the usage of \texttt{rbp}.
	
	Functions are called by \texttt{call} instruction, which pushes the next address into stack and jumpes. And \texttt{ret} to return, which pops and jumps to that value.
	
	\texttt{rbx}, \texttt{rsp}, \texttt{rbp}, \texttt{r12}, \texttt{r13}, \texttt{r14}, and \texttt{r15} are callee-saved registers. And \texttt{rax}, \texttt{rdi}, \texttt{rsi}, \texttt{rdx}, \texttt{rcx}, \texttt{r8}, \texttt{r9}, \texttt{r10}, \texttt{r11} are not preserved across function calls.
	
	The return value is stored in \texttt{rax}, and if the return value is 128-bit, \texttt{rdx} acts as the \nth{2} return register.
	
	There's a new concept called ``red zone" in this ABI. 128 bytes are subtracted from the stack before anything is pushed to the stack. It is the red zone, which should not be modified by signal or interrupt handlers. Functions can save temporary data which won't be needed across function calls here. 
	
	Same as the frame pointer, it is not a must - for example, Linux kernel won't respect the red zone, so when compiling Linux, argument \texttt{-mno-red-zone} is required.
	
	\subsection{My Question: How a Real-life AMD64 ELF Program Handles Variadic Function?}
	
	We try to assemble the following code.
	
	\begin{lstlisting}[language=c, caption={Example C code using variadic function}]
	#include <stdarg.h>
	#include <stdio.h>
	
	int sum(int cnt, ...) {
	    va_list ap;
	    int ret = 0;
	
	    va_start(ap, cnt); // locate ap
	
	    for (int i = 0; i < cnt; i++)
	        ret += va_arg(ap, int); // get arguments
	
	    va_end(ap); // cleanup
	
	    return ret;
	}
	\end{lstlisting}
	
	Assemble it with \texttt{gcc va.c -fno-stack-protector -S -shared}, we can get such assembly code.
	
	\begin{lstlisting}[language={[x64]Assembler}, caption={The assemble result}]
	.file	"va.c"
	.text
	.globl	sum
	.type	sum, @function
	sum:
	.LFB0:
		.cfi_startproc
		pushq	%rbp
		.cfi_def_cfa_offset 16
		.cfi_offset 6, -16
		movq	%rsp, %rbp
		.cfi_def_cfa_register 6
		subq	$104, %rsp
		movl	%edi, -212(%rbp)
		movq	%rsi, -168(%rbp)
		movq	%rdx, -160(%rbp)
		movq	%rcx, -152(%rbp)
		movq	%r8, -144(%rbp)
		movq	%r9, -136(%rbp)
		testb	%al, %al
		je	.L8
		movaps	%xmm0, -128(%rbp)
		movaps	%xmm1, -112(%rbp)
		movaps	%xmm2, -96(%rbp)
		movaps	%xmm3, -80(%rbp)
		movaps	%xmm4, -64(%rbp)
		movaps	%xmm5, -48(%rbp)
		movaps	%xmm6, -32(%rbp)
		movaps	%xmm7, -16(%rbp)
	.L8:
		movl	$0, -180(%rbp)
		movl	$8, -208(%rbp)
		movl	$48, -204(%rbp)
		leaq	16(%rbp), %rax
		movq	%rax, -200(%rbp)
		leaq	-176(%rbp), %rax
		movq	%rax, -192(%rbp)
		movl	$0, -184(%rbp)
		jmp	.L3
	.L6:
		movl	-208(%rbp), %eax
		cmpl	$47, %eax
		ja	.L4
		movq	-192(%rbp), %rax
		movl	-208(%rbp), %edx
		movl	%edx, %edx
		addq	%rdx, %rax
		movl	-208(%rbp), %edx
		addl	$8, %edx
		movl	%edx, -208(%rbp)
		jmp	.L5
	.L4:
		movq	-200(%rbp), %rax
		leaq	8(%rax), %rdx
		movq	%rdx, -200(%rbp)
	.L5:
		movl	(%rax), %eax
		addl	%eax, -180(%rbp)
		addl	$1, -184(%rbp)
	.L3:
		movl	-184(%rbp), %eax
		cmpl	-212(%rbp), %eax
		jl	.L6
		movl	-180(%rbp), %eax
		leave
		.cfi_def_cfa 7, 8
		ret
		.cfi_endproc
	.LFE0:
		.size	sum, .-sum
		.ident	"GCC: (GNU) 8.2.1 20181127"
		.section	.note.GNU-stack,"",@progbits
	
	\end{lstlisting}
	
	If all our arguments were stored on stack, this whole process would be quite easy to understand: \texttt{va\_start()} locates the address on stack, \texttt{va\_arg()} accesses the content on stack and jumps to the next argument, and \texttt{va\_end()} does the ``clean-up" work. 
	
	But in System V AMD64 ABI, as some arguments are stored in registers, we cannot implement a variadic library with plain C. In fact, \texttt{gcc} has some built-ins like \texttt{\_\_builtin\_va\_arg}, which gives the work to our compiler.
	
	After loading arguments to registers and stack, \texttt{sum} is \texttt{call}ed. In \texttt{sum}, arguments in registers are stored in memory. It's called ``Register Save Area" according to the document. In \texttt{.LFB0}, you may have noticed it checks whether \texttt{al} equals to 0, and if is, it will skip store contents in XMM registers. The \texttt{al} register actually stores the number of floating point numbers stored in registers. By checking its content, we can avoid copying XMM registers' values when our program is aimed at integers only.
	
	What's inside \texttt{va\_list}? It's now a structure rather than a simple pointer. 
	
	\begin{lstlisting}[language=c, caption={The implementation of \texttt{va\_list} in System V AMD64 ABI}]
	 typedef struct {
	   unsigned int gp_offset;
	   unsigned int fp_offset;
	   void *overflow_arg_area;
	   void *reg_save_area;
	 } va_list[1];
	\end{lstlisting}
	
	Here, \texttt{reg\_save\_area} points to the start of register save area, \texttt{overflow\_arg\_area} points to the start of arguments stores on stack, and is always updated to the next argument on stack. \texttt{gp\_offset} holds the offset of general purpose registers from \texttt{reg\_save\_area}, and \texttt{fp\_offset} holds that of floating point argument registers.
	
	In the assembly before, the \texttt{.L8} is mainly about \texttt{va\_start()}, which initialize that structure. The following part is about the loop and \texttt{va\_arg()}. It firstly checks whether register area is available, and take branches about different situations. For a simpler example about this part, see \cite[Figure 3.35]{amd64-abi}.
	
	What if the mechanism takes too many or too few arguments? Clean-up work is for caller, so takes too few arguments can never be a problem. But if callee takes too many arguments, it'll always get unwanted result. The difference is that if callee takes less than 7 arguments, it can never takes a segfault. However if it takes more than 6 arguments, it may access memory which is not allowed to read.
	
	And here it is! Lab 3 finished before 2019!
	
	\bibliography{ref}
\end{document}
